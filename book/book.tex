\documentclass[ebook, 9pt, openany]{memoir}
\usepackage{graphicx}
\renewcommand*\sfdefault{pag}
\renewcommand{\familydefault}{\sfdefault}
\setcounter{secnumdepth}{-1}
\setlength\beforechapskip{-60pt}
\setbeforesecskip{-1em}
\title{A Primer in Gont Poetics}
\author{Graduate Texts in Xenophilology}
\date{}

\begin{document}
\maketitle
\clearpage
\settocdepth{chapter}
\tableofcontents

\chapter{Some notes on the Gont language}

Since the original decipherment of the Gont script in the early 2210s, our
understanding of the Gont language has grown significantly. The following are
some brief notes intended as a refresher for the reader whose undergraduate
Gont classes are somewhat distant now. For a more thorough treatment, the
interested reader is directed to one of the many introductory texts available.

\section{Syntax}

Gont lacks any system of inflection, at least so far as is indicated in the
script. Sentences which contain a verb are ordered \emph{object-verb-subject}
(OVS). There is no evidence for the existence of intransitive verbs in Gont.
Sentences which lack a verb are ordered \emph{subject-predicate}.

\section{Script}

Gont phonology is largely unknown to us. For convenience, the components of
each glyph are transcribed with roman letters (consonants for Class A and
vowels for Class B), but these correspondences are entirely aribitrary. Each
glyph is transcribed clockwise, following the direction of the text as a whole.

Sequences of glyphs are read clockwise, spiralling out from the centre. In
literary Gont, it is standard to group glyphs in threes, which form a sequence
of expanding triangles within the text. In standard transcription, these are
rendered as lines of text. Each trigram reads clockwise, starting with the
uppermost glyph.

\section{Lexicon}

Those Gont glyphs whose meanings are well-understood are generally those used
in mathematical and scientific texts, the meanings of which can be verified
against our own scientific understanding. However, it is likely that many of
these glyphs are polysemic, and the technical meanings given here may not be
intended literally.\footnote{It is unlikely that a Gont would have the same
metaphorical uses of \emph{division} or \emph{pressure} as an English-speaker,
but similar usages must have existed.}


\include{book/poems}

\end{document}
